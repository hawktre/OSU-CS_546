% Options for packages loaded elsewhere
\PassOptionsToPackage{unicode}{hyperref}
\PassOptionsToPackage{hyphens}{url}
\PassOptionsToPackage{dvipsnames,svgnames,x11names}{xcolor}
%
\documentclass[
  letterpaper,
  DIV=11,
  numbers=noendperiod]{scrartcl}

\usepackage{amsmath,amssymb}
\usepackage{iftex}
\ifPDFTeX
  \usepackage[T1]{fontenc}
  \usepackage[utf8]{inputenc}
  \usepackage{textcomp} % provide euro and other symbols
\else % if luatex or xetex
  \ifXeTeX
    \usepackage{mathspec} % this also loads fontspec
  \else
    \usepackage{unicode-math} % this also loads fontspec
  \fi
  \defaultfontfeatures{Scale=MatchLowercase}
  \defaultfontfeatures[\rmfamily]{Ligatures=TeX,Scale=1}
\fi
\usepackage{lmodern}
\ifPDFTeX\else  
    % xetex/luatex font selection
\fi
% Use upquote if available, for straight quotes in verbatim environments
\IfFileExists{upquote.sty}{\usepackage{upquote}}{}
\IfFileExists{microtype.sty}{% use microtype if available
  \usepackage[]{microtype}
  \UseMicrotypeSet[protrusion]{basicmath} % disable protrusion for tt fonts
}{}
\makeatletter
\@ifundefined{KOMAClassName}{% if non-KOMA class
  \IfFileExists{parskip.sty}{%
    \usepackage{parskip}
  }{% else
    \setlength{\parindent}{0pt}
    \setlength{\parskip}{6pt plus 2pt minus 1pt}}
}{% if KOMA class
  \KOMAoptions{parskip=half}}
\makeatother
\usepackage{xcolor}
\usepackage[margin=0.75in]{geometry}
\setlength{\emergencystretch}{3em} % prevent overfull lines
\setcounter{secnumdepth}{-\maxdimen} % remove section numbering
% Make \paragraph and \subparagraph free-standing
\makeatletter
\ifx\paragraph\undefined\else
  \let\oldparagraph\paragraph
  \renewcommand{\paragraph}{
    \@ifstar
      \xxxParagraphStar
      \xxxParagraphNoStar
  }
  \newcommand{\xxxParagraphStar}[1]{\oldparagraph*{#1}\mbox{}}
  \newcommand{\xxxParagraphNoStar}[1]{\oldparagraph{#1}\mbox{}}
\fi
\ifx\subparagraph\undefined\else
  \let\oldsubparagraph\subparagraph
  \renewcommand{\subparagraph}{
    \@ifstar
      \xxxSubParagraphStar
      \xxxSubParagraphNoStar
  }
  \newcommand{\xxxSubParagraphStar}[1]{\oldsubparagraph*{#1}\mbox{}}
  \newcommand{\xxxSubParagraphNoStar}[1]{\oldsubparagraph{#1}\mbox{}}
\fi
\makeatother

\usepackage{color}
\usepackage{fancyvrb}
\newcommand{\VerbBar}{|}
\newcommand{\VERB}{\Verb[commandchars=\\\{\}]}
\DefineVerbatimEnvironment{Highlighting}{Verbatim}{commandchars=\\\{\}}
% Add ',fontsize=\small' for more characters per line
\usepackage{framed}
\definecolor{shadecolor}{RGB}{241,243,245}
\newenvironment{Shaded}{\begin{snugshade}}{\end{snugshade}}
\newcommand{\AlertTok}[1]{\textcolor[rgb]{0.68,0.00,0.00}{#1}}
\newcommand{\AnnotationTok}[1]{\textcolor[rgb]{0.37,0.37,0.37}{#1}}
\newcommand{\AttributeTok}[1]{\textcolor[rgb]{0.40,0.45,0.13}{#1}}
\newcommand{\BaseNTok}[1]{\textcolor[rgb]{0.68,0.00,0.00}{#1}}
\newcommand{\BuiltInTok}[1]{\textcolor[rgb]{0.00,0.23,0.31}{#1}}
\newcommand{\CharTok}[1]{\textcolor[rgb]{0.13,0.47,0.30}{#1}}
\newcommand{\CommentTok}[1]{\textcolor[rgb]{0.37,0.37,0.37}{#1}}
\newcommand{\CommentVarTok}[1]{\textcolor[rgb]{0.37,0.37,0.37}{\textit{#1}}}
\newcommand{\ConstantTok}[1]{\textcolor[rgb]{0.56,0.35,0.01}{#1}}
\newcommand{\ControlFlowTok}[1]{\textcolor[rgb]{0.00,0.23,0.31}{\textbf{#1}}}
\newcommand{\DataTypeTok}[1]{\textcolor[rgb]{0.68,0.00,0.00}{#1}}
\newcommand{\DecValTok}[1]{\textcolor[rgb]{0.68,0.00,0.00}{#1}}
\newcommand{\DocumentationTok}[1]{\textcolor[rgb]{0.37,0.37,0.37}{\textit{#1}}}
\newcommand{\ErrorTok}[1]{\textcolor[rgb]{0.68,0.00,0.00}{#1}}
\newcommand{\ExtensionTok}[1]{\textcolor[rgb]{0.00,0.23,0.31}{#1}}
\newcommand{\FloatTok}[1]{\textcolor[rgb]{0.68,0.00,0.00}{#1}}
\newcommand{\FunctionTok}[1]{\textcolor[rgb]{0.28,0.35,0.67}{#1}}
\newcommand{\ImportTok}[1]{\textcolor[rgb]{0.00,0.46,0.62}{#1}}
\newcommand{\InformationTok}[1]{\textcolor[rgb]{0.37,0.37,0.37}{#1}}
\newcommand{\KeywordTok}[1]{\textcolor[rgb]{0.00,0.23,0.31}{\textbf{#1}}}
\newcommand{\NormalTok}[1]{\textcolor[rgb]{0.00,0.23,0.31}{#1}}
\newcommand{\OperatorTok}[1]{\textcolor[rgb]{0.37,0.37,0.37}{#1}}
\newcommand{\OtherTok}[1]{\textcolor[rgb]{0.00,0.23,0.31}{#1}}
\newcommand{\PreprocessorTok}[1]{\textcolor[rgb]{0.68,0.00,0.00}{#1}}
\newcommand{\RegionMarkerTok}[1]{\textcolor[rgb]{0.00,0.23,0.31}{#1}}
\newcommand{\SpecialCharTok}[1]{\textcolor[rgb]{0.37,0.37,0.37}{#1}}
\newcommand{\SpecialStringTok}[1]{\textcolor[rgb]{0.13,0.47,0.30}{#1}}
\newcommand{\StringTok}[1]{\textcolor[rgb]{0.13,0.47,0.30}{#1}}
\newcommand{\VariableTok}[1]{\textcolor[rgb]{0.07,0.07,0.07}{#1}}
\newcommand{\VerbatimStringTok}[1]{\textcolor[rgb]{0.13,0.47,0.30}{#1}}
\newcommand{\WarningTok}[1]{\textcolor[rgb]{0.37,0.37,0.37}{\textit{#1}}}

\providecommand{\tightlist}{%
  \setlength{\itemsep}{0pt}\setlength{\parskip}{0pt}}\usepackage{longtable,booktabs,array}
\usepackage{calc} % for calculating minipage widths
% Correct order of tables after \paragraph or \subparagraph
\usepackage{etoolbox}
\makeatletter
\patchcmd\longtable{\par}{\if@noskipsec\mbox{}\fi\par}{}{}
\makeatother
% Allow footnotes in longtable head/foot
\IfFileExists{footnotehyper.sty}{\usepackage{footnotehyper}}{\usepackage{footnote}}
\makesavenoteenv{longtable}
\usepackage{graphicx}
\makeatletter
\newsavebox\pandoc@box
\newcommand*\pandocbounded[1]{% scales image to fit in text height/width
  \sbox\pandoc@box{#1}%
  \Gscale@div\@tempa{\textheight}{\dimexpr\ht\pandoc@box+\dp\pandoc@box\relax}%
  \Gscale@div\@tempb{\linewidth}{\wd\pandoc@box}%
  \ifdim\@tempb\p@<\@tempa\p@\let\@tempa\@tempb\fi% select the smaller of both
  \ifdim\@tempa\p@<\p@\scalebox{\@tempa}{\usebox\pandoc@box}%
  \else\usebox{\pandoc@box}%
  \fi%
}
% Set default figure placement to htbp
\def\fps@figure{htbp}
\makeatother

\KOMAoption{captions}{tablesignature}
\usepackage{fancyhdr, amsthm, amssymb,amsfonts,amsthm, amsmath, bbm}
\usepackage{float, tabularx}
\floatplacement{table}{H}
\pagestyle{fancy}
\fancyhead[R]{Notebook 11}
\fancyhead[L]{Trent VanHawkins}
\fancyfoot[C]{\thepage}
\makeatletter
\@ifpackageloaded{caption}{}{\usepackage{caption}}
\AtBeginDocument{%
\ifdefined\contentsname
  \renewcommand*\contentsname{Table of contents}
\else
  \newcommand\contentsname{Table of contents}
\fi
\ifdefined\listfigurename
  \renewcommand*\listfigurename{List of Figures}
\else
  \newcommand\listfigurename{List of Figures}
\fi
\ifdefined\listtablename
  \renewcommand*\listtablename{List of Tables}
\else
  \newcommand\listtablename{List of Tables}
\fi
\ifdefined\figurename
  \renewcommand*\figurename{Figure}
\else
  \newcommand\figurename{Figure}
\fi
\ifdefined\tablename
  \renewcommand*\tablename{Table}
\else
  \newcommand\tablename{Table}
\fi
}
\@ifpackageloaded{float}{}{\usepackage{float}}
\floatstyle{ruled}
\@ifundefined{c@chapter}{\newfloat{codelisting}{h}{lop}}{\newfloat{codelisting}{h}{lop}[chapter]}
\floatname{codelisting}{Listing}
\newcommand*\listoflistings{\listof{codelisting}{List of Listings}}
\makeatother
\makeatletter
\makeatother
\makeatletter
\@ifpackageloaded{caption}{}{\usepackage{caption}}
\@ifpackageloaded{subcaption}{}{\usepackage{subcaption}}
\makeatother

\usepackage{bookmark}

\IfFileExists{xurl.sty}{\usepackage{xurl}}{} % add URL line breaks if available
\urlstyle{same} % disable monospaced font for URLs
\hypersetup{
  pdftitle={CS546 Class Notebook 12 - co-analyzing the human PPI network and Gene Ontology `biological process' annotations},
  pdfauthor={Trent VanHawkins},
  colorlinks=true,
  linkcolor={blue},
  filecolor={Maroon},
  citecolor={Blue},
  urlcolor={Blue},
  pdfcreator={LaTeX via pandoc}}


\title{CS546 Class Notebook 12 - co-analyzing the human PPI network and
Gene Ontology `biological process' annotations}
\author{Trent VanHawkins}
\date{2026-02-19}

\begin{document}
\maketitle


In this notebook exercise, we will (inspired by Figure 1C in Rhodes et
al.) compute the likelihood ratio (LR) for human protein-protein pairs
to have an interaction (vs.~not have an interaction), given the size
(i.e., number of genes) of the smallest (i.e., most specific) Gene
Ontology biological process term that the two proteins share in common.
To do this, we will be co-analyzing the human protein-protein
interaction network from Pathway Commons and protein annotations (Gene
Ontology biological process) from UniprotKB. Your analysis will produce
empirical estimates of the LR for interacting/not-interacting, as a
function of the size \texttt{size\_smallest\_shared\_bp\_int} of the
smallest GO biological process shared by a given protein pair; note, as
in the Rhodes et al.~paper, this will be done in a discrete binning of
the \texttt{size\_smallest\_shared\_bp\_int} value, given limitations on
the data set size.

Import the python module dependencies for this notebook: pandas as pd,
numpy as np, random, and pprint

\begin{Shaded}
\begin{Highlighting}[]
\ImportTok{import}\NormalTok{ pandas }\ImportTok{as}\NormalTok{ pd}
\ImportTok{import}\NormalTok{ numpy }\ImportTok{as}\NormalTok{ np}
\ImportTok{import}\NormalTok{ random}
\ImportTok{import}\NormalTok{ pprint}
\end{Highlighting}
\end{Shaded}

From S3, download the Pathway Commons database and uncompress it

\begin{Shaded}
\begin{Highlighting}[]
\CommentTok{\#!curl https://csx46.s3{-}us{-}west{-}2.amazonaws.com/PathwayCommons9.All.hgnc.sif.gz \textbackslash{}}
      \CommentTok{\#{-}{-}output PathwayCommons9.All.hgnc.sif.gz}
\end{Highlighting}
\end{Shaded}

Using the exclamation point to run \texttt{gunzip} at the Linux shell,
uncompress the file \texttt{PathwayCommons9.All.hgnc.sif.gz}, producing
an uncompressed file \texttt{PathwayCommons9.All.hgnc.sif}.

\begin{Shaded}
\begin{Highlighting}[]
\CommentTok{\#!gunzip {-}f PathwayCommons9.All.hgnc.sif.gz}
\end{Highlighting}
\end{Shaded}

From S3, download the UniprotKB proteome database (a tab-separated file
of protein information)

\begin{Shaded}
\begin{Highlighting}[]
\OperatorTok{!}\NormalTok{curl https:}\OperatorTok{//}\NormalTok{csx46.s3.us}\OperatorTok{{-}}\NormalTok{west}\OperatorTok{{-}}\FloatTok{2.}\ErrorTok{amazonaws}\NormalTok{.com}\OperatorTok{/}\NormalTok{uniprotkb\_proteome\_UP000005640\_2024\_01\_29.tsv }\OperatorTok{\textgreater{}}\NormalTok{ ..}\OperatorTok{/}\NormalTok{DataRaw}\OperatorTok{/}\NormalTok{uniprotkb\_proteome\_UP000005640\_2024\_01\_29.tsv}
\end{Highlighting}
\end{Shaded}

\begin{verbatim}
  % Total    % Received % Xferd  Average Speed   Time    Time     Time  Current
                                 Dload  Upload   Total   Spent    Left  Speed
100 24.7M  100 24.7M    0     0  21.6M      0  0:00:01  0:00:01 --:--:-- 21.6M
\end{verbatim}

Using Linux shell tools \texttt{head}, \texttt{sed}, and \texttt{nl},
take a peek at the first row of the file, to get an enumerated list of
the column names. Don't forget to use an exclamation point so that your
line of code is run in the Linux shell and not in the python
interpreter.

\begin{Shaded}
\begin{Highlighting}[]
\OperatorTok{!}\NormalTok{head }\OperatorTok{{-}}\DecValTok{1}\NormalTok{ ..}\OperatorTok{/}\NormalTok{DataRaw}\OperatorTok{/}\NormalTok{uniprotkb\_proteome\_UP000005640\_2024\_01\_29.tsv }\OperatorTok{|}\NormalTok{ sed }\StringTok{\textquotesingle{}s/}\CharTok{\textbackslash{}t}\StringTok{/}\CharTok{\textbackslash{}n}\StringTok{/g\textquotesingle{}} \OperatorTok{|}\NormalTok{ nl}
\end{Highlighting}
\end{Shaded}

\begin{verbatim}
     1  Entry
     2  Reviewed
     3  Entry Name
     4  Protein names
     5  Gene Names
     6  Organism
     7  Length
     8  Gene Ontology (cellular component)
     9  Gene Ontology (biological process)
    10  Gene Ontology (molecular function)
\end{verbatim}

Load the Pathway Commons SIF file \texttt{PathwayCommons9.All.hgnc.sif}
into a Pandas dataframe \texttt{sif\_data}, renaming the three columns
\texttt{species1}, \texttt{interaction\_type}, and \texttt{species2}

\begin{Shaded}
\begin{Highlighting}[]
\NormalTok{sif\_data }\OperatorTok{=}\NormalTok{ pd.read\_csv(}\StringTok{"../DataRaw/PathwayCommons9.All.hgnc.sif"}\NormalTok{,}
\NormalTok{                       sep}\OperatorTok{=}\StringTok{"}\CharTok{\textbackslash{}t}\StringTok{"}\NormalTok{, names}\OperatorTok{=}\NormalTok{[}\StringTok{"species1"}\NormalTok{,}
                                        \StringTok{"interaction\_type"}\NormalTok{,}
                                        \StringTok{"species2"}\NormalTok{])}
\end{Highlighting}
\end{Shaded}

process the protein-protein interaction data to eliminate duplicates

\begin{Shaded}
\begin{Highlighting}[]
\NormalTok{interaction\_types\_ppi }\OperatorTok{=} \BuiltInTok{set}\NormalTok{([}\StringTok{"interacts{-}with"}\NormalTok{,}
                             \StringTok{"in{-}complex{-}with"}\NormalTok{])}
\NormalTok{interac\_ppi }\OperatorTok{=}\NormalTok{ sif\_data[sif\_data.interaction\_type.isin(interaction\_types\_ppi)].copy()}
\NormalTok{inds\_swap }\OperatorTok{=}\NormalTok{ interac\_ppi[}\StringTok{\textquotesingle{}species1\textquotesingle{}}\NormalTok{] }\OperatorTok{\textgreater{}}\NormalTok{ interac\_ppi[}\StringTok{\textquotesingle{}species2\textquotesingle{}}\NormalTok{]}
\NormalTok{interac\_ppi.loc[inds\_swap, [}\StringTok{\textquotesingle{}species1\textquotesingle{}}\NormalTok{, }\StringTok{\textquotesingle{}species2\textquotesingle{}}\NormalTok{]] }\OperatorTok{=}\NormalTok{ interac\_ppi.loc[inds\_swap, [}\StringTok{\textquotesingle{}species2\textquotesingle{}}\NormalTok{, }\StringTok{\textquotesingle{}species1\textquotesingle{}}\NormalTok{]].values}
\NormalTok{interac\_ppi\_unique }\OperatorTok{=}\NormalTok{ interac\_ppi[[}\StringTok{"species1"}\NormalTok{, }\StringTok{"species2"}\NormalTok{]].drop\_duplicates()}
\end{Highlighting}
\end{Shaded}

load a tab-separated data file
\texttt{uniprotkb\_proteome\_UP000005640\_2024\_01\_29.tsv} of protein
information into a dataframe \texttt{prot\_data}, retaining only columns
\texttt{Gene\ Names} and \texttt{Gene\ Ontology\ (biological\ process)}

\begin{Shaded}
\begin{Highlighting}[]
\NormalTok{prot\_data }\OperatorTok{=}\NormalTok{ pd.read\_csv(}\StringTok{"../DataRaw/uniprotkb\_proteome\_UP000005640\_2024\_01\_29.tsv"}\NormalTok{, sep}\OperatorTok{=}\StringTok{"}\CharTok{\textbackslash{}t}\StringTok{"}\NormalTok{)[[}\StringTok{"Gene Names"}\NormalTok{, }\StringTok{"Gene Ontology (biological process)"}\NormalTok{]]}
\end{Highlighting}
\end{Shaded}

make a dictionary \texttt{gene\_names\_dict} whose key is a gene name,
and whose value is a list of Gene Ontology biological process annotation
terms for that gene (fill in the steps below as indicated in the code
comments with \texttt{TODO})

\begin{Shaded}
\begin{Highlighting}[]
\NormalTok{ctr }\OperatorTok{=} \DecValTok{0}
\NormalTok{gene\_names\_dict }\OperatorTok{=} \BuiltInTok{dict}\NormalTok{()}

\CommentTok{\# }\AlertTok{TODO}\CommentTok{: using a \textasciigrave{}for\textasciigrave{} loop,}
\CommentTok{\# iterate over prot\_data using the \textasciigrave{}iterrows\textasciigrave{} method,}
\CommentTok{\# in each iteration assigning the row index to \textasciigrave{}index\textasciigrave{} and}
\CommentTok{\# the tuple of values of the row of the dataframe, to \textasciigrave{}row\textasciigrave{}}
\ControlFlowTok{for}\NormalTok{ index, row }\KeywordTok{in}\NormalTok{ prot\_data.iterrows():}
\NormalTok{    ctr }\OperatorTok{+=} \DecValTok{1}
\NormalTok{    gene\_names\_str }\OperatorTok{=}\NormalTok{ row.iloc[}\DecValTok{0}\NormalTok{]}
    \ControlFlowTok{if} \BuiltInTok{isinstance}\NormalTok{(gene\_names\_str, }\BuiltInTok{float}\NormalTok{) }\KeywordTok{and}\NormalTok{ np.isnan(gene\_names\_str):}
        \ControlFlowTok{continue}
\NormalTok{    gene\_names\_list }\OperatorTok{=}\NormalTok{ []}
    \CommentTok{\# }\AlertTok{TODO}\CommentTok{ split the string \textasciigrave{}gene\_names\textasciigrave{} by instances of the space (\textasciigrave{} \textasciigrave{})}
    \CommentTok{\# character, and assign the first (i.e., 0th in python indexing)}
    \CommentTok{\# substring to the varible name \textasciigrave{}gene\textasciigrave{}}
\NormalTok{    gene }\OperatorTok{=}\NormalTok{ gene\_names\_str.split(}\StringTok{\textquotesingle{} \textquotesingle{}}\NormalTok{)[}\DecValTok{0}\NormalTok{]}
    \ControlFlowTok{if} \StringTok{\textquotesingle{}{-}\textquotesingle{}} \KeywordTok{in}\NormalTok{ gene:}
        \CommentTok{\# sometimes, the canonical gene name is a pair of}
        \CommentTok{\# official gene symbols with a hyphen ("{-}") between them;}
        \CommentTok{\# need to handle this case using an inelegant hack:}
        \ControlFlowTok{if} \BuiltInTok{all}\NormalTok{(}\BuiltInTok{len}\NormalTok{(gn) }\OperatorTok{\textgreater{}} \DecValTok{3} \ControlFlowTok{for}\NormalTok{ gn }\KeywordTok{in}\NormalTok{ gene\_names\_str.split(}\StringTok{\textquotesingle{}{-}\textquotesingle{}}\NormalTok{)):}
            \ControlFlowTok{for}\NormalTok{ gn }\KeywordTok{in}\NormalTok{ gene\_names\_str.split(}\StringTok{\textquotesingle{}{-}\textquotesingle{}}\NormalTok{):}
\NormalTok{                gene\_names\_list.append(gn)}
        \ControlFlowTok{else}\NormalTok{:}
\NormalTok{            gene\_names\_list.append(gene)}
    \ControlFlowTok{else}\NormalTok{:}
\NormalTok{        gene\_names\_list.append(gene)}
\NormalTok{    go\_bp }\OperatorTok{=}\NormalTok{ row.iloc[}\DecValTok{1}\NormalTok{]}
    \ControlFlowTok{if} \KeywordTok{not} \BuiltInTok{isinstance}\NormalTok{(go\_bp, }\BuiltInTok{float}\NormalTok{) }\KeywordTok{or} \KeywordTok{not}\NormalTok{ np.isnan(go\_bp):}
        \CommentTok{\# }\AlertTok{TODO}\CommentTok{: split up the string \textasciigrave{}go\_bp\textasciigrave{} by occurrences}
        \CommentTok{\# of the separator character \textasciigrave{}";"\textasciigrave{}, and for each}
        \CommentTok{\# resulting substring, strip whitespace using the}
        \CommentTok{\# \textasciigrave{}string.strip()\textasciigrave{} instance method and add it to}
        \CommentTok{\# a final list \textasciigrave{}go\_bp\_list\textasciigrave{}}
\NormalTok{        go\_bp\_list }\OperatorTok{=}\NormalTok{ go\_bp.split(}\StringTok{\textquotesingle{};\textquotesingle{}}\NormalTok{)}
\NormalTok{        go\_bp\_list }\OperatorTok{=}\NormalTok{ [bp.strip() }\ControlFlowTok{for}\NormalTok{ bp }\KeywordTok{in}\NormalTok{ go\_bp\_list]}
    \ControlFlowTok{else}\NormalTok{:}
\NormalTok{        go\_bp\_list }\OperatorTok{=}\NormalTok{ []}
    \ControlFlowTok{for}\NormalTok{ gene\_name }\KeywordTok{in}\NormalTok{ gene\_names\_list:}
\NormalTok{        gene\_names\_dict[gene\_name] }\OperatorTok{=}\NormalTok{ go\_bp\_list}
\end{Highlighting}
\end{Shaded}

starting with \texttt{gene\_names\_dict}, make a data frame
\texttt{go\_df} containing just two columns, \texttt{gene} and
\texttt{bp}, where \texttt{gene} contains the gene symbol and
\texttt{bp} contains a list of GO biological process term annotations
for that gene symbol

\begin{Shaded}
\begin{Highlighting}[]
\NormalTok{go\_df }\OperatorTok{=}\NormalTok{ pd.DataFrame(\{}\StringTok{\textquotesingle{}gene\textquotesingle{}}\NormalTok{: gene\_names\_dict.keys(),}
                      \StringTok{\textquotesingle{}bp\textquotesingle{}}\NormalTok{: gene\_names\_dict.values()\})}
\NormalTok{go\_df.head(n}\OperatorTok{=}\DecValTok{10}\NormalTok{)}
\end{Highlighting}
\end{Shaded}

\begin{longtable}[]{@{}lll@{}}
\toprule\noalign{}
& gene & bp \\
\midrule\noalign{}
\endhead
\bottomrule\noalign{}
\endlastfoot
0 & DMD & {[}{]} \\
1 & DGKI & {[}{]} \\
2 & BOLA2 & {[}protein maturation by iron-sulfur cluster tra... \\
3 & SMG1P6 & {[}protein maturation by iron-sulfur cluster tra... \\
4 & CYP2D7 & {[}{]} \\
5 & PTGS1 & {[}response to oxidative stress {[}GO:0006979{]}{]} \\
6 & HNF1A & {[}{]} \\
7 & PTPN22 & {[}{]} \\
8 & PIGBOS1 & {[}{]} \\
9 & SIK1B & {[}intracellular signal transduction {[}GO:0035556... \\
\end{longtable}

using \texttt{pandas.Series.to\_dict()},
\texttt{pandas.DataFrame.groupby},
\texttt{pandas.DataFrameGroupBy.apply}, and \texttt{np.sum}, make a
dictionary \texttt{gene\_to\_go} relating the canonical gene name to the
list of GO biological process term annotations for that gene

\begin{Shaded}
\begin{Highlighting}[]
\NormalTok{gene\_to\_go }\OperatorTok{=}\NormalTok{ pd.Series.to\_dict(go\_df.groupby([go\_df.gene]).bp.}\BuiltInTok{apply}\NormalTok{(np.}\BuiltInTok{sum}\NormalTok{))}
\end{Highlighting}
\end{Shaded}

make a dictionary \texttt{go\_to\_gene} relating GO biological process
terms to genes that are annotated with the GO biological process term in
the key (this is known as ``inverting'' a dictionary); so, the key in
this dictionary will be a Gene Ontology (GO) term and the value will be
a Python list of gene symbols; it may be convenient to use a double for
loop (outer loop is over key/value pairs \texttt{g}, \texttt{tl}
returned by \texttt{gene\_to\_go.items()}, and the inner loop is over
list elements of the \texttt{tl} list)

\begin{Shaded}
\begin{Highlighting}[]
\NormalTok{go\_to\_gene }\OperatorTok{=} \BuiltInTok{dict}\NormalTok{()}
\ControlFlowTok{for}\NormalTok{ g, tl }\KeywordTok{in}\NormalTok{ gene\_to\_go.items():}
    \ControlFlowTok{for}\NormalTok{ t }\KeywordTok{in}\NormalTok{ tl:}
\NormalTok{        go\_to\_gene[t] }\OperatorTok{=}\NormalTok{ go\_to\_gene.get(t, []) }\OperatorTok{+}\NormalTok{ [g]}
\end{Highlighting}
\end{Shaded}

calculate, for all pairs of interacting proteins (mapped to gene names),
the size of the smallest shared GO biological process annotation for the
genes (fill in the steps below as indicated in the code comments with
\texttt{TODO})

\begin{Shaded}
\begin{Highlighting}[]
\NormalTok{size\_smallest\_shared\_bp\_int }\OperatorTok{=}\NormalTok{ []}
\NormalTok{no\_shared\_bp\_int }\OperatorTok{=} \DecValTok{0}
\NormalTok{int\_set }\OperatorTok{=} \BuiltInTok{set}\NormalTok{()  }\CommentTok{\# we will need a set of "keys" of interacting proteins; we will use it later}
\ControlFlowTok{for}\NormalTok{ row }\KeywordTok{in}\NormalTok{ interac\_ppi\_unique.iterrows():}
\NormalTok{    g1 }\OperatorTok{=}\NormalTok{ row[}\DecValTok{1}\NormalTok{].species1}
\NormalTok{    g2 }\OperatorTok{=}\NormalTok{ row[}\DecValTok{1}\NormalTok{].species2}
\NormalTok{    int\_set.add(g1 }\OperatorTok{+} \StringTok{\textquotesingle{}{-}\textquotesingle{}} \OperatorTok{+}\NormalTok{ g2)}
\NormalTok{    go1 }\OperatorTok{=} \BuiltInTok{set}\NormalTok{(gene\_to\_go.get(g1, []))}
\NormalTok{    go2 }\OperatorTok{=} \BuiltInTok{set}\NormalTok{(gene\_to\_go.get(g2, []))}
\NormalTok{    go12\_terms }\OperatorTok{=}\NormalTok{ go1 }\OperatorTok{\&}\NormalTok{ go2}
    \ControlFlowTok{if} \BuiltInTok{len}\NormalTok{(go12\_terms) }\OperatorTok{\textgreater{}} \DecValTok{0}\NormalTok{:}
        \CommentTok{\# }\AlertTok{TODO}\CommentTok{: use dictionary comprehension to process the set \textasciigrave{}go12\_terms\textasciigrave{} to}
        \CommentTok{\# create a dictionary \textasciigrave{}go12\_terms\_sizes\textasciigrave{} mapping GO biological process terms}
        \CommentTok{\# \textasciigrave{}t\textasciigrave{} from the set, to values defined by \textasciigrave{}len(go\_to\_gene[t])\textasciigrave{}:}
\NormalTok{        go12\_terms\_sizes }\OperatorTok{=}\NormalTok{ \{t: }\BuiltInTok{len}\NormalTok{(go\_to\_gene[t]) }\ControlFlowTok{for}\NormalTok{ t }\KeywordTok{in}\NormalTok{ go12\_terms\}}
        \CommentTok{\# }\AlertTok{TODO}\CommentTok{: using the python builtin \textasciigrave{}min\textasciigrave{} with the optional function}
        \CommentTok{\# argument \textasciigrave{}key\textasciigrave{}, find the key (which we\textquotesingle{}ll assign to \textasciigrave{}min\_term\textasciigrave{})}
        \CommentTok{\# in \textasciigrave{}go12\_terms\_sizes\textasciigrave{} whose associated value is the smallest of any}
        \CommentTok{\# value in that dictionary; assign that value to \textasciigrave{}size\_min\_term\textasciigrave{}}
\NormalTok{        min\_term }\OperatorTok{=} \BuiltInTok{min}\NormalTok{(go12\_terms\_sizes, key}\OperatorTok{=}\NormalTok{go12\_terms\_sizes.get)}
\NormalTok{        size\_min\_term }\OperatorTok{=}\NormalTok{ go12\_terms\_sizes[min\_term]}
        \CommentTok{\# }\AlertTok{TODO}\CommentTok{: append \textasciigrave{}size\_min\_term\textasciigrave{} to the list \textasciigrave{}size\_smallest\_shared\_bp\_int\textasciigrave{}}
\NormalTok{        size\_smallest\_shared\_bp\_int.append(size\_min\_term)}
    \ControlFlowTok{else}\NormalTok{:}
\NormalTok{        no\_shared\_bp\_int }\OperatorTok{+=} \DecValTok{1}
\end{Highlighting}
\end{Shaded}

calculate, for ten million random pairs of \emph{non-interacting}
proteins (mapped to gene names), the size of the smallest shared GO
biological process annotation for the genes

\begin{Shaded}
\begin{Highlighting}[]
\NormalTok{size\_smallest\_shared\_bp\_no\_int }\OperatorTok{=}\NormalTok{ []}
\NormalTok{no\_shared\_bp\_no\_int }\OperatorTok{=} \DecValTok{0}
\NormalTok{all\_genes }\OperatorTok{=} \BuiltInTok{list}\NormalTok{(gene\_to\_go.keys())}
\NormalTok{ctr }\OperatorTok{=} \DecValTok{0}
\NormalTok{Nnoint }\OperatorTok{=} \DecValTok{10000000}
\ControlFlowTok{while}\NormalTok{ ctr }\OperatorTok{\textless{}}\NormalTok{ Nnoint:}
\NormalTok{    g1 }\OperatorTok{=}\NormalTok{ random.choice(all\_genes)}
\NormalTok{    g2 }\OperatorTok{=}\NormalTok{ g1}
    \ControlFlowTok{while}\NormalTok{ g2 }\OperatorTok{==}\NormalTok{ g1 }\KeywordTok{or}\NormalTok{ (g1 }\OperatorTok{+} \StringTok{\textquotesingle{}{-}\textquotesingle{}} \OperatorTok{+}\NormalTok{ g2) }\KeywordTok{in}\NormalTok{ int\_set:  }\CommentTok{\# use the "key" to check if they are interacting}
\NormalTok{        g2 }\OperatorTok{=}\NormalTok{ random.choice(all\_genes)}
\NormalTok{    go1 }\OperatorTok{=} \BuiltInTok{set}\NormalTok{(gene\_to\_go.get(g1, []))}
\NormalTok{    go2 }\OperatorTok{=} \BuiltInTok{set}\NormalTok{(gene\_to\_go.get(g2, []))}
\NormalTok{    go12\_terms }\OperatorTok{=}\NormalTok{ go1 }\OperatorTok{\&}\NormalTok{ go2}
    \ControlFlowTok{if} \BuiltInTok{len}\NormalTok{(go12\_terms) }\OperatorTok{\textgreater{}} \DecValTok{0}\NormalTok{:}
\NormalTok{        go12\_terms\_sizes }\OperatorTok{=}\NormalTok{ \{t: }\BuiltInTok{len}\NormalTok{(go\_to\_gene[t]) }\ControlFlowTok{for}\NormalTok{ t }\KeywordTok{in}\NormalTok{ go12\_terms\}}
\NormalTok{        min\_term }\OperatorTok{=} \BuiltInTok{min}\NormalTok{(go12\_terms\_sizes, key}\OperatorTok{=}\NormalTok{go12\_terms\_sizes.get)}
\NormalTok{        size\_min\_term }\OperatorTok{=}\NormalTok{ go12\_terms\_sizes[min\_term]}
\NormalTok{        size\_smallest\_shared\_bp\_no\_int.append(size\_min\_term)}
    \ControlFlowTok{else}\NormalTok{:}
\NormalTok{        no\_shared\_bp\_no\_int }\OperatorTok{+=} \DecValTok{1}
\NormalTok{    ctr }\OperatorTok{+=} \DecValTok{1}
\end{Highlighting}
\end{Shaded}

use Numpy's histogram feature to calculate the likelihood ratios using
the same binning based on GO biological process gene-set size that
Rhodes et al.~used (fill in the steps below as indicated in the code
comments with \texttt{TODO})

\begin{Shaded}
\begin{Highlighting}[]
\NormalTok{breaks }\OperatorTok{=}\NormalTok{ [}\DecValTok{0}\NormalTok{, }\DecValTok{5}\NormalTok{, }\DecValTok{10}\NormalTok{, }\DecValTok{50}\NormalTok{, }\DecValTok{100}\NormalTok{, }\DecValTok{500}\NormalTok{]}
\NormalTok{Nint }\OperatorTok{=}\NormalTok{ interac\_ppi\_unique.shape[}\DecValTok{0}\NormalTok{]}
\NormalTok{l\_no\_shared }\OperatorTok{=}\NormalTok{ (no\_shared\_bp\_int }\OperatorTok{/}\NormalTok{ Nint)}\OperatorTok{/}\NormalTok{(no\_shared\_bp\_no\_int }\OperatorTok{/}\NormalTok{ Nnoint)}
\NormalTok{hist\_int }\OperatorTok{=}\NormalTok{ np.histogram(size\_smallest\_shared\_bp\_int, bins}\OperatorTok{=}\NormalTok{breaks)}
\CommentTok{\# }\AlertTok{TODO}\CommentTok{: using \textasciigrave{}np.histogram\textasciigrave{}, create a histogram \textasciigrave{}host\_no\_int\textasciigrave{} from}
\CommentTok{\# \textasciigrave{}size\_smallest\_shared\_bp\_no\_int\textasciigrave{} using the set of breaks \textasciigrave{}breaks\textasciigrave{},}
\CommentTok{\# similarly to how we created \textasciigrave{}hist\_int\textasciigrave{}}
\NormalTok{hist\_no\_int }\OperatorTok{=}\NormalTok{ np.histogram(size\_smallest\_shared\_bp\_no\_int, bins}\OperatorTok{=}\NormalTok{breaks)}
\CommentTok{\# }\AlertTok{TODO}\CommentTok{: define list \textasciigrave{}l\_ratios\textasciigrave{} as the ratio \textasciigrave{}(hist\_int[0])/Nint\textasciigrave{} over}
\CommentTok{\# \textasciigrave{}(hist\_no\_int[0]/Nnoint)\textasciigrave{}}
\NormalTok{l\_ratios }\OperatorTok{=}\NormalTok{ (hist\_int[}\DecValTok{0}\NormalTok{]}\OperatorTok{/}\NormalTok{Nint) }\OperatorTok{/}\NormalTok{ (hist\_no\_int[}\DecValTok{0}\NormalTok{]}\OperatorTok{/}\NormalTok{Nnoint)}
\NormalTok{l\_ratios\_res }\OperatorTok{=}\NormalTok{ [(}\StringTok{\textquotesingle{}no relation\textquotesingle{}}\NormalTok{, l\_no\_shared)]}
\ControlFlowTok{for}\NormalTok{ ctr }\KeywordTok{in} \BuiltInTok{range}\NormalTok{(}\BuiltInTok{len}\NormalTok{(l\_ratios)):}
\NormalTok{    l\_ratios\_res.append((}\SpecialStringTok{f"}\SpecialCharTok{\{}\NormalTok{breaks[ctr]}\SpecialCharTok{\}}\SpecialStringTok{{-}}\SpecialCharTok{\{}\NormalTok{breaks[ctr}\OperatorTok{+}\DecValTok{1}\NormalTok{]}\SpecialCharTok{\}}\SpecialStringTok{"}\NormalTok{,}
                         \BuiltInTok{float}\NormalTok{(l\_ratios[ctr])))}
\NormalTok{pprint.pprint(l\_ratios\_res)}
\end{Highlighting}
\end{Shaded}

\begin{verbatim}
[('no relation', 0.972249385351156),
 ('0-5', 29.31661841985238),
 ('5-10', 22.954393168271768),
 ('10-50', 15.207707470157905),
 ('50-100', 12.536113599391244),
 ('100-500', 5.819699431895828)]
\end{verbatim}




\end{document}
